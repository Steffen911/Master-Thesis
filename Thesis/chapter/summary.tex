\chapter{Summary and Future Work}
\label{ch:summary}

In this Chapter, we will review the main contributions and results of this Thesis.
Afterwards, we will look at possible future extensions of the work presented here.

In Chapter~\ref{ch:relevant-concepts} we gave a general introduction into the Semantic Web and the reasons behind
its creation.
We showed how annotations in the HTML-code can ensure that content that was created for humans becomes
readable for machines.
We also introduced the ideas behind ontology matching and, a special case of ontology matching, taxonomy matching.
Taxonomy matching presents itself as especially useful in an e-commerce environments to enhance catalog integration
tasks and product searches.

Relevant contributions in the field of the Semantic Web, ontology matching, and product taxonomy matching are presented
in Chapter~\ref{ch:related-work}.
In  addition to  a description of the relevant methods, we designed a categorization system for product taxonomy matching
algorithms in which we integrated the methods from Chapter~\ref{ch:related-work} and the ones used throughout this Thesis.

To compare different methods for product taxonomy matching, we created a training dataset from product information
that was crawled from the Semantic Web.
We found that labels derived from shared instances have an overall low quality and are insufficient for reliable
testing and, therefore, we manually annotated a subset of this training set to create a product
taxonomy matching gold standard.
The  result is a  set of class-label pairs from different e-commerce platforms that are labelled as either \emph{equal}, \emph{contains},
\emph{contained-in}, or \emph{disjoint}.

We described the methods that we test on this gold standard in Chapter~\ref{ch:taxonomy-matching}.
This includes three major types of methods.
The first group consists of static methods that only use the information given in the two class-labels that should be
labelled.
The algorithms in the second group are also static in the sense that they do not benefit from training data, but they include
external corpora like WordNet to enhance their results.
The last group consists of machine learning models that are trained on our training set and use the two class-labels
as an input to make their predictions.

Chapter~\ref{ch:experiment-results} dealt with the results of our experiments.
For each method, we provided the precision, recall, and F1-score per positive label and the corresponding confusion
matrix.

In Chapter~\ref{ch:error-analysis} we analysed those results and tried to identify explanations for the phenomena
we observed in the experiments.
In doing so, we presented individual examples for class-label pairs that are either hard or easy to predict for a given algorithm and
provide recommendations for cases where a certain algorithm may prove useful.
Overall, we noticed that product taxonomy matching is a tough problem and no method provides fully satisfying
results.
Especially the added complexity of more advanced models is often not justified by a significant improvement over the
baseline methods.

The implementations we have used in order to replicate the results of relevant contributions on a real-world dataset
follow the description of the authors as close as possible and we used open source packages where available.
Nevertheless, we had to add small adjustments  to cater to our requirement of providing semantic annotations, i.e., detection
of equality or if one class is more general than another.
Only our machine learning models and S-Match~\cite{giunchiglia2005semantic} provide this functionality out of the box.
However, those adjustments do not explain the overall bad performance of the models under consideration.

Thus, as a part of this Thesis, we have identified two fields that deserve additional attention.

First, all models across the literature require that the two taxonomies under consideration use similar concepts to
label their products.
We agree that this is a necessary requirement to detect equality or another semantic relationship, but we also found classes
that share a set of products, but use orthogonal classification methods.
An example may be a classification either by sport (Sports $>$ Golf $>$ Polo Shirts) or by brand (Sports $>$ Under Armour $>$ Polo Shirts).
The two class-labels would certainly share some products, but do not fit any of the conventional semantic labels.
Those partial overlaps can not be handled by any model that we are aware of.
We detected those in our dataset based on shared instances, but it would be interesting to classify them based on the
class-label alone.

The second area is the semantic labelling of class-label pairs that we also covered in this Thesis.
Most existing methods focus on finding the best match for a given class-label in a target taxonomy.
While this certainly has a use case for catalog integration, the more generic solution could also enhance product
search tasks and possibly more topics.

In conclusion, we have seen that product taxonomy matching is a relevant problem with room for improvement.
On  the one hand, we provided insights into the specific problem one faces with product taxonomies and,
on the other hand, assist further research with the gold standard we created.
