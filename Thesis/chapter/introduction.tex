\chapter{Introduction and Motivation}
\label{ch:introduction}

Taxonomy matching, a particular case of ontology matching, is a recurring topic among researchers.
Numerous novel approaches are tested in this domain and some achieve surprising results.
Yet, most of them are applied to artificial datasets that do not represent the real world.
In general, the goal of taxonomy matching is to create an alignment between two distinct categorization systems, e.g.,
the taxonomy of a library and a book shop.
Although they may sell the same products, they probably use different approaches to catalog them.
The outputs of the taxonomy matching algorithm create a semantic bridge between the labels used by both stores.

The Ontology Alignment Evaluation Initiative\footnote{\url{http://oaei.ontologymatching.org}. Accessed: 01.05.2020} (OAEI)
exists since 2004 and publishes an ever-increasing number of datasets that can be used for the evaluation
and comparison of ontology alignment algorithms.
Contributions leveraging those datasets are numerous and are presented at the International
Semantic Web Conference\footnote{\url{https://iswc2020.semanticweb.org}. Accessed: 01.05.2020}.
The existence of the OAEI establishes a need for large, real-world ontology matching datasets
that challenge researchers and enable the development of new ontology alignment methods.

The Web Data Commons\footnote{\url{http://www.webdatacommons.org}. Accessed: 01.05.2020} (WDC)
project shows that an  increasing number of e-commerce websites annotate their pages semantically
using schema.org and other shared vocabularies.
These make a given webpage understandable to a machine and enable comparability of information across
different domains.
As a result, a large corpus of product information with annotated categories/classes that embody
real-world taxonomies is available to everyone.
Moreover, the Semantic Web offers a vast corpus for the ontology matching  task.
Each  website may use its own  dedicated taxonomy.
This presents us with a possibly large, real-world dataset that we can use for the evaluation of taxonomy
matching algorithms.

Avesani et al.\@~\cite{avesani2005large} as well as Angerman and Ramzan~\cite{angermann2017taxonomy} identify the
necessity of large-scale taxonomy matching datasets that represent the real-world.
They state that most researchers use small or artificial datasets and that there is no agreed-upon reference dataset.
Angerman and Ramzan explicitly express the need for freely available, large-scale datasets.

Hence, it is the goal of this Thesis to investigate if the semantic annotations can be leveraged to create an
evaluation dataset for ontology matching tasks that stems from publicly available data.
In this Thesis, we will create a gold standard of labelled class pairs that are part of multiple,
distinct, real-world taxonomies and evaluate numerous taxonomy matching algorithms on this gold standard.
The alignment tasks that the OAEI focuses on are mainly concerned with closest match problems.
Instead, we build on the ideas of Giunchiglia et al.\@~\cite{giunchiglia2005semantic} and provide semantic labels.
They predict not only equality of classes, but also if one class is more general than the other.
This could enable improvements in product search on pages that feature results from multiple distinct pages,
e.g., price-comparison sites, by enabling customers to broaden or focus their search more efficiently.
It could also help in the task of catalog integration, as discussed in Meusel et al.\@~\cite{meusel2015exploiting}
and Sabou et al.\@~\cite{sabou2008exploring}.
Instead of putting all products under the closest class in a target taxonomy, the need for an additional
layer may become apparent and, therefore, improve the overall result.
We also review and categorize relevant literature with a special focus on e-commerce taxonomy matching.
Our main contributions are:
\begin{itemize}
    \item an overview of relevant taxonomy matching literature,
    \item the creation of a realistic gold standard for semantic taxonomy matching in the e-commerce domain, and
    \item an evaluation of product taxonomy matching algorithms on a real-world dataset.
\end{itemize}

The rest of this Thesis is structured as follows.
Chapter~\ref{ch:relevant-concepts} introduces the fundamentals of this Thesis.
Hence, the Semantic Web and taxonomy matching are defined.
It  also definitions that we will use throughout this Thesis.
We review relevant contributions from current research in Chapter~\ref{ch:related-work}.
There, we present contributions in the field of the Semantic Web and different taxonomy matching approaches.
In the fourth Chapter, the creation of our gold standard is outlined.
We describe the methods that are evaluated on the gold standard in Chapter~\ref{ch:taxonomy-matching}
and evaluate the results in Chapter~\ref{ch:experiment-results}.
Chapter~\ref{ch:error-analysis} includes an analysis of the errors that individual methods make in the prediction.
Finally, Chapter~\ref{ch:summary} summarizes our results and gives an outlook into possible
future work.
